\documentclass[11pt]{article}
\usepackage{graphicx}
\usepackage{amssymb}
%\usepackage[nomarkers]{endfloat}
\usepackage{natbib}
\usepackage{setspace}
\usepackage{wasysym}
\usepackage{wrapfig}

\textwidth = 6.5 in
\textheight = 9 in
\oddsidemargin = 0.0 in
\evensidemargin = 0.0 in
\topmargin = 0.0 in
\headheight = 0.0 in
\headsep = 0.0 in
\parskip = 0.1in
\parindent = 0in

\begin{document}
\pagestyle{empty}

\begin{center}
{\bf \large BIOL 701 Likelihood Methods in Biology:  Homework \#1 Answers}
\end{center}
1). In a predation experiment, 5 different prey items are in arena.  A predator consumes two items, one after the other.  How many outcomes in
\newline
a). Five choices for the first item, and 4 choices for the second, so there are 20 possible outcomes.
\newline
b). There are four choices for the second if the first item consumed is \#4, so there are 4 possible outcomes.
\newline
c).  Similarly, there are four choices for the first if the second item consumed is \#4, so there are 4 possible outcomes
\newline
d). This is the sum of b and c since they are mutually exclusive, 8 possible outcomes.
\newline
c). Four choices for the first item and 3 choices for the second, so there are 12 possible outcomes.
\vspace*{.1in}

2).  The binoculars of four ornithologists (A, B, C, and D) get mixed up in a bag when running away from a charging rhino.  After they escape, binoculars are passed out at random upon hearing a Blue-throated Barbet.  
\newline
a).  There are 4!=24 different ways to pass out the binoculars at random.  If C gets her own, there are 3!=6 ways to pass out the remaining 3 pairs, so the probability is $\frac{3!}{4!}=\frac{1}{4}$.
\newline
b).  If A and D get their own binoculars, there are 2! ways to distribute the remaining two pairs, so the probability is $\frac{2!}{4!}=\frac{1}{12}$.
\newline
c).  If B, C and D get their own binoculars, there is only 1 way to distribute the remaining pair, so the probability is $\frac{1!}{4!}=\frac{1}{24}$.


but they are all trampled or gored because they were only paying attention to birds so it's academic.
\vspace*{.1in}

3).  Let $\Omega$ be the set of points with positive integer coordinates:  $\omega=(i,j)$, and define a probability distribution by $p(\omega)=1/2^{i+j}$.

a).  $\sum p(\omega)=\sum_{i=1}^\infty \sum_{j=1}^\infty \frac{1}{2^{i+j}}=\sum_{i=1}^\infty \sum_{j=1}^\infty \frac{1}{2^{i}}\frac{1}{2^{j}}=\frac{1}{2}\sum_{j=1}^\infty \frac{1}{2^{j}}+\frac{1}{4}\sum_{j=1}^\infty \frac{1}{2^{j}}+\cdots$.

We know that $\sum_{j=0}^\infty \frac{1}{2^{j}}=\frac{1}{1-\frac{1}{2}}=2$, which implies that $\sum_{j=1}^\infty \frac{1}{2^{j}}=\sum_{j=0}^\infty \frac{1}{2^{j}}-\frac{1}{2^0}=1$, which means that each sum over j above equals one and the whole expression reduces to a sum over i, which we know equals 1 from the geometric series above.

b).  Probability of the event $\{(i,j): i+j\leq 4\}=\sum_{i=1}^3\sum_{j=1}^{4-i}=\frac{1}{4}+\frac{1}{8}+\frac{1}{16}+\frac{1}{8}+\frac{1}{16}+\frac{1}{16}=\frac{11}{16}$
\vspace*{.1in}

4).  If you assume that the sex of each child is determined independently, then observing one of Smith's children gives you no information about the sex of the other.  

The second line of reasoning might seem valid if we are randomly selecting ``families," pairs of children, at random from the sample space.  In that case, knowing that one of the children in the pair is a boy does rule out the GG family, but the question asks what is the probability that the second child is a boy.  We could assume that each family \{BB, BG, GB, GG\} has a prior probability of 1/4 and then ask what the probability of each family type is given that we met a boy.  

First we need $P(B)=P(B|BB)P(BB)+P(B|BG)P(BG)+P(B|GB)P(GB)+P(B|GG)P(GG)=1*\frac{1}{4}+\frac{1}{2}*\frac{1}{4}+\frac{1}{2}*\frac{1}{4}+0*\frac{1}{4}=\frac{1}{2}$.  This is an example of conditioning on a partition of the sample space.

Then $P(BB|B)=P(B|BB)P(BB)/P(B)=\frac{1}{4}/\frac{1}{2}=\frac{1}{2}$.

\vspace*{.1in}

5).  $(X, Y)$ has the bivariate CDF $F(x,y)=\frac{1}{2}(x^2y+xy^2)$ on the unit square:  $0<x<1$, $0<y<1$.

a).  $F(1,1)=\frac{1}{2}(1+1)=1$

b).  $f(x,y)=\frac{\partial ^2F(x,y)}{\partial x\partial y}=\frac{\partial }{\partial y}\frac{1}{2}(2xy+y^2)=\frac{1}{2}(2x+2y)=x+y$

c).  $F(x, y>1)=\frac{1}{2}(x^2+x)$ since $P(Y\leq a)$, for $a>1$ is 1.  Y can only assume values less than 1 so the probability that Y is less than a number greater than 1 is 1.

d).  $f_X(x)=\int_0^1f(x,y)dy=\int_0^1(x+y)dy=(xy+\frac{y^2}{2})\bigg|_{y=0}^{y=1}=x+\frac{1}{2}$.  By symmetry, $f_Y(y)=y+\frac{1}{2}$.

e).  $P(X+Y<1)=\int_o^1\int_0^{1-y}(x+y)dxdy=\int_0^1\frac{(1-y)^2}{2}+(1-y)ydy=\int_0^1\frac{1}{2}-\frac{y^2}{2}dy=\frac{1}{3}$

\vspace*{.1in}
6). The joint p.d.f of $(X,Y)$ is $f(x,y)=6(1-x-y)$ for $0<y<1-x$ and $0<x<1$.  Find the conditional p.d.f.'s $f(x|y)$ and $f(y|x)$.

$f_X(x)=\int_0^{1-x}6(1-x-y)dy=6(y-xy-\frac{y^2}{2})\bigg|_{y=0}^{y=1-x}=6(\frac{1}{2}-x+\frac{x^2}{2})=3(1-x)^2$

$f(y|x)=\frac{f(x,y)}{f_X(x)}=\frac{2(1-x-y)}{(1-x)^2}$

$f_Y(y)=\int_0^{1-y}6(1-x-y)=3(1-y)^2 \rightarrow f(x|y)=\frac{2(1-x-y)}{(1-y)^2}$



\end{document}