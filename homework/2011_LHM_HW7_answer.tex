\documentclass[11pt]{article}
\usepackage{graphicx}
\DeclareGraphicsRule{.tif}{png}{.png}{`convert #1 `dirname #1`/`basename #1 .tif`.png}

\textwidth = 6.5 in
\textheight = 9 in
\oddsidemargin = 0.0 in
\evensidemargin = 0.0 in
\topmargin = 0.0 in
\headheight = 0.0 in
\headsep = 0.0 in
\parskip = 0.2in
\parindent = 0.0in
\usepackage{paralist} %compactenum
\usepackage{bm}
%\newtheorem{theorem}{Theorem}
%\newtheorem{corollary}[theorem]{Corollary}
%\newtheorem{definition}{Definition}
\usepackage{tipa}
\usepackage{amsfonts}
\usepackage[mathscr]{eucal}

% Use the natbib package for the bibliography
\usepackage[round]{natbib}
\bibliographystyle{apalike}
\newcommand{\answer}[1]{{\color{red}\sc Begin Answer:\\}#1{\par\color{red}\sc End Answer.\\}}

\title{Brief Article}
\author{The Author}

\usepackage{url}
\usepackage{hyperref}
\hypersetup{backref,  pdfpagemode=FullScreen,  linkcolor=blue, citecolor=red, colorlinks=true, hyperindex=true}

\begin{document}
\begin{center}
	{\bf BIOL 701  Likelihood Methods in Biology: Homework \#7}\\
	{Due Wednesday, May 11th} 
\end{center}
\subsection*{Background}
You are interested in fecundity of an annual plant (as assessed by the total mass of seeds produced measured in g) as a function of fertilization regimes.
You have collected data from several years in eight different fields.  
Within each field season, you examined plants four different treatments in each field.
These treatments correspond to growing the plants with (coded as 1) and without (coded as 0) fertilization with a nitrogen source and a phosphorus source.
Thus, the four treatments are: no fertilization, +N, +P, and +N,P.
For each treatment$\times$field$\times$year combination, you recorded data for four plants.

You are interested in inferring the effectiveness of different treatments in general (e.g. the expected effect of adding Nitrogen to some unspecified field), as well as learning about which fields have the highest yield.

Consider a model in which each of the following effects contribute additively to the expected mass for an individual:
\begin{compactenum}
	\item a year effect (centered around 0),
	\item a field-specific expected mass without fertilization (this effect is the same for a field across all years),
	\item a field-specific effect of nitrogen fertilization (this effect is the same for a field across all years),
	\item a field-specific effect of phosphorus fertilization (this effect is the same for a field across all years)
	\item a field-specific effect of adding both N and P (this effect is the same for a field across all years).
\end{compactenum}
Of course, you should also expect some variability around this expected value (not all individuals from the same treatment, field, and year will have exactly the same mass).

Note that you would like to make some general conclusions about the effect of different fertilizer treatments on some hypothetical field.
So, you should structure the model so that you can learn about expected effects across fields, while accounting for the fact that their may be variability in the response of a particular field to a particular treatment.

\subsection*{Tasks}
\begin{compactenum}
	\item Write down the likelihood for your model.
	\\\answer{
	$i$ indexes the year\\
	$j$ indexes the field\\
	$k$ indexes the Nitrogen (0= no Nitrogen, 1 = Nitrogen)\\
	$m$ indexes the Phosphorus (0= no Phosphorus, 1 = Phosphorus) \\
	$q$ indexes the individual in that year$\times$field$\times$treatment\\
	$$y_{ijkmq} \sim N(A_i + B_j + kG_j + mD_j + kmR_j, \sigma_e^2)$$
	$a$ is the number of years\\
	$b$ is the number of fields\\
	$n_{ijkm}$ is the number of individuals in the specified year$\times$field$\times$treatment\\
	$$f(y_{ijkmq}|A_i, B_j, G_j, D_j, R_j, \sigma_e) = \frac{1}{\sqrt{2\pi,\sigma_e}}e^{\frac{-(y_{ijkmq} - A_i - B_j - kG_j - mD_j - kmR_j)^2}{2\sigma_e^2}}$$
	$$f(\bm Y|\bm A, \bm B,\bm G,\bm D,\bm R, \sigma_e) = \prod_{i=1}^a \prod_{j=1}^b\prod_{k=0}^1\prod_{m=0}^1\prod_{q=1}^{n_{ijkm}} f(y_{ijkmq}|A_i, B_j, G_j, D_j, R_j, \sigma_e) $$
	}
	\item List each parameter and the prior distribution that you have chosen to use for the parameter.
	\\\answer{ (there are lots of possible ways to answer - priors are up to the person analyzing the data!

\begin{center}
\begin{tabular}{|p{9cm}l|}
\hline
The standard deviation of the year effects &  $\sigma_a^2 \sim \mbox{Exponential}(\lambda=.1)$ \\
\hline
$a$ year effects (latent variables)&  $A_i\sim \mbox{Normal}(0, \sigma_a^2)$\\
\hline
The expected value of the field effects (no fert)&  $\mu_b \sim \mbox{Gamma}(\mbox{mean}=250, \mbox{variance}=50)$\\
\hline
The standard deviation of the field effects(no fert) &  $\sigma_b^2 \sim \mbox{Exponential}(\lambda=.1)$\\
\hline
$b$ field effects (latent variables)&  $B_j\sim \mbox{Normal}(\mu_b, \sigma_b^2)$\\
\hline
The expected value of the nitrogen effects&  $\mu_g \sim \mbox{Normal}(\mbox{mean}=5, \mbox{variance}=10)$\\
\hline
The standard deviation of the nitrogen effects&  $\sigma_g^2 \sim \mbox{Exponential}(\lambda=.1)$\\
\hline
$b$ field$\times$nitrogen effects (latent variables)&  $G_j\sim \mbox{Normal}(\mu_g, \sigma_g^2)$\\
\hline
The expected value of the phosphorus effects&  $\mu_d \sim \mbox{Normal}(\mbox{mean}=5, \mbox{variance}=10)$\\
\hline
The standard deviation of the phosphorus effects&  $\sigma_d^2 \sim \mbox{Exponential}(\lambda=.1)$\\
\hline
$b$ field$\times$phosphorus effects (latent variables)&  $D_j\sim \mbox{Normal}(\mu_d, \sigma_d^2)$\\
\hline
The expected value of the nitrogen$\times$phosphorus interaction effect&  $\mu_r \sim \mbox{Normal}(\mbox{mean}=0, \mbox{variance}=10)$\\
\hline
The standard deviation of the nitrogen$\times$phosphorus interaction effects&  $\sigma_r^2 \sim \mbox{Exponential}(\lambda=.1)$\\
\hline
$b$ field$\times$nitrogen$\times$phosphorus interaction effects (latent variables)&  $R_j\sim \mbox{Normal}(\mu_r, \sigma_r^2)$\\
\hline
The environmental/error starndard deviation&  $\sigma_e^2\sim \mbox{Exponential}(\lambda=.1)$\\
\hline
\end{tabular}
\end{center}
\label{default}
	}
	\item Implement an MCMC algorithm that uses the data in \href{http://phylo.bio.ku.edu/slides/fertilization_data.csv}{fertilization\_data.csv} to inference posterior distributions for the parameters. \href{http://phylo.bio.ku.edu/slides/latent_gekko_svl.py.txt}{latent\_gekko\_svl.py.txt} may be useful as a template. Email me your implementation.
	\item Perform an MCMC simulation. Summarize the evidence that your MCMC run has been conducted for enough iterations (\href{http://tree.bio.ed.ac.uk/software/tracer/}{Tracer} or \href{http://cran.r-project.org/web/packages/coda/index.html}{CODA} may be helpful for this) to generate useful results.
	\item Based on your runs, answer the following questions (and give a brief explanation of how you calculate the answers).
	\begin{compactenum}
		\item If you could scale the experiment up to an unlimited number of fields, what is the probability that the mean effect of adding nitrogen alone will be an effect of 5g or greater?
		\item Which field has the largest expected mass without fertilization?  
		\item What is the probability is the probability that field \# 6 is the most productive (has the highest expected mass over a large number of years) if both N and P are added?
\end{compactenum}
		
\end{compactenum}

\end{document}  

